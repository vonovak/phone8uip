\chapter{Introduction}
Spatial navigation and mobility is an integral part of every day living. Whether a person wants to go on a field trip, go shopping or visit a physician, they need to be able to navigate themselves. For certain groups of people, however, independent exercise of these actions is harder than for others. These include especially visually and motor impaired, elderly, or people with other disabilities.\\
With today's development of technology, there are new means of helping these groups to navigate in various environments without being dependent on another person's help. Development of navigational solutions has been of interest of both commercial and academic research groups and with the availability of GPS, there were great advancements in various fields of outdoor usages ranging from navigation of individuals on roads and in cities to agriculture and marine or aviation applications.\\
Still, indoor navigation, where GPS is not available, remains a not-so-developed part of the field. Various tools are used to navigate people indoors, such as maps and visual navigation systems (including banners, signs on the floor, flashing light and etc.). These systems, however, assume that their user does not suffer from visual impairment and, generally, has a good health.\\Indeed, research shows that for a healthy person, sight accounts for 70-90\% of information that the brain receives \cite{hyerle}. Visually impaired therefore lack the most important information source and have to use their other senses and assistive tools to navigate and orient themselves. It should be noted that there are situations when not only visually impaired have problems to orientate themselves. In large and complex buildings, such as hospitals, airports, university or government facilities and other, navigating can be a tough task even for a fit and able person. 

\section{Motivation}
The World Health Organization states there are 285 million people estimated to be visually impaired worldwide, of whom 39 million are blind and 246 have low vision. Also, the number of elderly people - who often suffer some form of motor or other impairment - is growing. This is true especially in the developed economies of western Europe, USA or Japan and according to the US Census Bureau report \cite{uscensus}, the number elderly people (citizens 65-years-old and older) in the United States is expected to double within the next four decades, making up 21\% of the US population.\\The data shows that the target-group for an indoor navigation system is growing and technologies that assist with indoor navigation have perspective.
To help tackling the problem, there has been a number of research projects proposing solutions to indoor navigation \cite{naviterier}, \cite{percept}, \cite{lopez}, \cite{luis}, \cite{riehle} of visually or motor impaired or elderly. One such project focused on visually impaired, called NaviTerier is being developed at the FEE CTU. The project, described in greater detail in section \ref{sec:nsa} is intended to run on a handheld device (a smartphone) and works on the principle of sequential presentation of carefully prepared descriptions of the building segments to visually impaired user by means of mobile phone voice output. UIProtocol (UIP) - another research project of FEE CTU has been used together with NaviTerier to form NUIP which combines the navigation part of NaviTerier and the ability to generate user interfaces of UIP. NUIP consist of several sub-systems: firstly, there is a route planner which supports customizing of the route according to the user abilities and preferences. Secondly,  a navigation description generator which creates description of the route with respect to the users limitations was developed. Finally, a context-sensitive UI generator \cite{macik2} adapts the user interface according to navigation terminals and personal devices used during the navigation.\\

For this system to be available to as wide a set of users we need to have UIP clients for the most common smartphone platforms. So far, UIP clients are built for Windows, iOS and Android. Since there is no client for the Windows Phone 8 (WP8), within this thesis we intend to build upon the NUIP project and bring an UIProtocol client application to this platform. 

\section{Assignment}
The goal of this thesis is to develop an accessible UIProtocol client for Windows Phone 8. To do so, we will analyze the UIProtocol and introduce the reader to its features and architecture. Since the developed client is intended to work in conjunction with the NaviTerier project we will also cover the state of the art of indoor navigation systems.\\The platform of development was chosen ahead of time, however, we will perform an accessibility analysis of the most common smartphone platforms to be able to compare the platform's friendliness toward disabled users and put it into context of other mobile platforms.\\The main part is the actual development of the UIProtocol client which will be covered extensively. The last task is to verify the functionality of the implemented client, for which we developed a testing application todo unit tests.

\section{Outline of the Document}
The document has five more chapters:

\begin{description}
  \item[Chapter 2] \hfill \\
  Contains analysis which was carried out before the development of the client. The analysis covers the UIProtocol itself, discusses the accessibility of today's smartphone platforms, shows other projects in the field of indoor navigation and shortly explains the accessibility guidelines for WP8 development.
  \item[Chapter 3] \hfill \\
  Goes through the analysis of the UIProtocol client. We cover the the outcomes of the analysis, requirements for the app and how we designed the application architecture. Several UML diagrams are included.
  \item[Chapter 4] \hfill \\
  Establishing onto the analysis, the fourth chapter covers the implementation of UIProtocol client, problems encountered during the development and how they were solved.
  \item[Chapter 5] \hfill \\
  To finish the development cycle, testing of the developed application is needed. For this, a testing application and todo unit test were developed. This chapter cover the results of testing.
    \item[Chapter 6] \hfill \\
    The last chapter discusses the conclusions and future work.
\end{description}
