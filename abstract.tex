\startAbstractCz
UIProtocol je jazyk pro specifikaci uživatelských rozhraní, jež je vyvíjen na FEL ČVUT pro účely výzkumu. Je navržen jako klient-server systém, kde server má k dispozici soubory popisující aplikaci a její uživatelské rozhraní. Tento popis je poskytnut klientu, který vykresluje ono uživatelské rozhraní, informuje server o akcích uživatele a zpracovává odpovědi serveru na tyto události. Tato práce popisuje vývoj přístupného UIProtocol klienta pro platformu Windows Phone 8. Tento klient je vyvýjen pod projektem NaviTerier UIP (NUIP), jehoý cílem je vyvinout systém pro navigaci nevidomích uvnitř budov. Tato práce se proto bude zabývat i způsoby, jak zvýšit přístupnost aplikace pro hendikepované uživatele, například zrakově postižené. Klient musí být schopen provádět standardní úkony, jako komunikace se serverem, vykreslování uživatelských rozhraní, obsluha akcí uživatele a další.
\stopAbstractCz

\startAbstractEn
UIProtocol is a user interface specification language being developed at FEE CTU for research purposes. It is designed as a client-server system where the server runs an app and provides its user interface description to the client. The client renders the user interface, informs the server of user-triggered events and processes the server response. This work describes the development of an accessible UIProtocol client for Windows Phone 8 platform. The client is developed under the NaviTerier UIP (NUIP) project at FEE CTU, which aims to create a system for navigating visually impaired inside buildings. Therefore we will explore the possibilities of making the client more accessible to people with disabilities, such as visually impaired. The client has to be able to perform standard operations such as communicating with the server, rendering the user interfaces, handling user actions and more.
\stopAbstractEn

\endinput
%%
%% End of file `abstract.tex'.
