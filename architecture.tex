\chapter{Design}
This chapter will go through the design of the application architecture of our UIP Client. The A



\subsection{Requirements}
The application should perform well in terms of performance. There will be a delay present in the application's reaction time, which is a consequence of the client-server architecture ie. the need of internet connection. The delay should be reasonably small to allow for a comfortable usage of the application.
\\
The application should run on any device powered by the Windows Phone 8 OS.
\newline
\\
Non-functional requirements:
\begin{itemize}
  \item UI components have platform native look
  \item App will be will be written in C\#
  \item The client should not use much phone resources when idle
  \item App should be stable and able to process valid UIP documents
  \item Compatibility with UIP standard, draft 8
  \item UI loading times below 0.5 s with stable connections
\end{itemize}
~\\
Functional requirements:
\begin{itemize}
  \item  Support for basic user interface elements
  \item Graceful degradation for unsupported elements
  \item  Support for binding and model-wide binding
  \item  Support for interpolated model updates (animations)
    \item  Support for UI generator API
  \item Support for absolute and grid layouts
    \item Support for styling
\end{itemize}

\subsection{Design}
The design of the application is what determines its maintainability and ability for future upgrades and adding new functionality. We therefore put a great deal of emphasis on making the code well structured and extendable.

\subsubsection{Overview of Core Classes}
In this section, we will cover the most important classes of the application, to give a brief idea of how the UIP documents are handled, stored, processed and how the UI is rendered. There are several tables in the following pages, illustrating how inner UIP Document representation is stored (Table \ref{tab:uipDocClasses}), how rendering functions (Table \ref{tab:uipRenderClasses}) and how the communication with the server is handled. Apart from the classes mentioned in the tables, there is a number of other helper classes.


\begin{table}[htbp]
  \centering
  \caption{UIP Document Representation Classes}
  \label{tab:uipDocClasses}
 \renewcommand{\arraystretch}{1.2}
    \begin{tabularx}{\textwidth}{p{2.5cm}|X}
    \rowcolor{mygray}
    \textbf{Class Name} & \textbf{Class Description} \\
       Interface & This class represents the UIP interface as a container for more UI elements. This class has its own position, a container and can be embedded into another interface, as specified in Listing \ref{uipInterface}. \\ \hline
       Container & This is the class that stores the information about the particular UI elements. A Container can contain other Containers and instances of Element class.\\ \hline
       Element & Class representing particular UI elements such as button, textfield and more. \\
    \end{tabularx}%
    \label{tab:uipDocClasses2}
\end{table}%

\begin{table}[htbp]
  \centering
  \caption{Classes ensuring the rendering of UI elements}
  \label{tab:uipRenderClasses}
 \renewcommand{\arraystretch}{1.2}
    \begin{tabularx}{\textwidth}{p{3cm}|X}
    \rowcolor{mygray}
    \textbf{Class name} & \textbf{Class description} \\
       Renderer & The main class responsible for rendering the elements stored in the classes of table \ref{tab:uipDocClasses}. Its rendering method walks through the tree structure of UIP Document and invokes rendering of each element. It also does the graceful degradation of unsupported elements. \\ \hline
       IRenderable & An interface which defines methods for acquiring class, style, position and other properties of UIP elements. It is implemented by all classes in table \ref{tab:uipDocClasses}. \\ \hline
       \hspace{0pt}IRenderableContainer & Extension of \texttt{IRenderable} interface. It provides support for layouts and is implemented by instances of Interface and Container. \\
    \end{tabularx}%
\end{table}%

\begin{table}[htbp]
  \centering
  \caption{UIP Server Connection Classes}
  \label{tab:uipCommClasses}
 \renewcommand{\arraystretch}{1.2}
    \begin{tabularx}{\textwidth}{p{2.5cm}|X}
    \rowcolor{mygray}
    \textbf{Class Name} & \textbf{Class Description} \\
       UipConnection & Initiates the connection and is responsible for sending events to the server and processing its responses. Does basic XML validation. \\ \hline
       SocketWorker & Handles the socket communication with UIP server. Sends events and runs a separate thread for receiving server's responses. \\
    \end{tabularx}%
\end{table}%

\subsubsection{Communication With UIP Server}
As mentioned in Table TODO, the communication with server in implemented in UipConnection class which exposes its functionality for sending events to the rest of the application - namely the EventManager class. It also is responsible for any XML data received from server and this functionality is used by the SocketWorker class.
\\
SocketWorker class sends events to the server, such as interface requests or events informing about user actions. It also runs an instance of BackgroundWorker class which awaits data from the server. The reason there is a separate thread for received data is that server can decide to send data at any time, not only as a response to a certain user action.

Model Updates and Binding\\
