
\subsection{Android Accessibility}
Similarly to the previous section, here we will analyze the accessibility options for devices running the Android operating system, with the emphasis on visually impaired users. We will analyze the features of the latest Android OS released, which is version 4.4, code name KitKat. It should be noted that there were no major updates to the accessibility options since Android 4.2.2 Jelly Bean.

\subsubsection{Speech Features}
Android also offers the option to interact with the device using speech and has some interesting accessibility features and compared to Windows Phone 8 offers a wider language support. Similarly to Windows Phone, an Android developer can take advantage of speech recognition and text-to-speech (TTS). Android comes with a number of built-in voice commands but unlike the Windows Phone, Android does not allow developers to expose their own voice commands. The last important feature on Android is TalkBack.
At the time of writing, the speech recognition supports more than 40 languages including several accents of English, and even minor languages such as Czech. Other functions do not have such a wide support. 
\paragraph{Speech Recognition}
Users can give input to an app or accomplish tasks with it using speech recognition. An example usage can be dictating content of an SMS. As mentioned before, this feature supports many languages but on the other hand required internet connection. There is not an option to use the recognition offline. We do not consider this a major drawback, as nowadays a mobile internet connection is more available than ever.
\paragraph{Voice Action Commands}
In Android, Voice Action Commands are closely related to the Google Now feature. Google Now has a wide range of uses not specifically designed for visually impaired. It can, however, serve them well by allowing them to get information using voice.
In general, google now should provide the user with relevant information when they need it. Google describes it by the phrase “The right information at just the right time”. This includes telling the user the weather forecast, showing the best route to work, calling someone, creating a reminder and much more. The full list of Voice action Commands is in Table \ref{tab:asc}.

Note that for some commands, the system gives you a spoken answer. The current drawback of the system is that it only supports English, French, German, Spanish, and Italian. With other languages, user can only make a voice-induced Google search with no voice response.
\paragraph{Text to Speech (TTS)}
TTS can be used to speak text to the user via the phone's speaker or headset. The spoken text can be simple strings. The industry-standard Speech Synthesis Markup Language (SSML) is supported only in a limited scope. TTS is also used in TalkBack which is described in the next section.
\paragraph{Other Speech Features}
TalkBack is an important functionality that strives for more accessible phone control for visually impaired. Basically, it is a touch-controlled screen reader. When enabled, user can drag finger across the screen selecting the components and getting their acoustic description. By double tapping anywhere in the screen, user can open/use the last selected item. TalkBack also supports gestures. This way, a user can get a complete description of the user interface. The blog post of a blind accessibility engineer from Mozilla Foundation \cite{mozillaguy} claims that visually impaired users of this system still have to overcome some obstacles.

\subsubsection{Other Tools for Ease of Access}
Android too comes with more features for ease of access which can help lightly visually impaired users which include change font size and the screen magnifier.

\subsubsection{Conclusions}
Compared to Windows Phone, Android has better accessibility options. It includes the usual functions, such as text to speech, speech recognition or font size settings. It also offers a built-in screen reader, called TalkBack. Android aims to be usable even for visually impaired.

