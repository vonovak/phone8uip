\chapter{Analysis}
Before moving onto the design and implementation chapters of the work, allow us to introduce you to the technology used for development, analysis of UIProtocol, overview of other navigation systems and accessibility analysis of the current mobile platforms.


\section{About Windows Phone 8}
The Windows Phone 8 is the first of Microsoft's mobile platform to use the Windows NT Kernel, which is the same kernel as the one in Windows 8 \cite{wp8kernel}. Therefore some parts of the API are the same for both systems. A significant subset of Windows Runtime is built natively into Windows Phone 8, with the functionality exposed to all supported languages. This gives a developer the ability to use the same API for common tasks such as networking, working with sensors, processing location data, and implementing in-app purchase. Therefore there is more potential for code reuse.\\
Also, Windows Phone 8 and windows 8 share the same .NET engine \cite{wp8comparison}. This is to deliver more stability and performance to the apps, so they can take advantage of multicore processing and improve battery life. Most new devices are now multicore, and the operating system and apps are expected to be faster because of this technology \cite{wp8comparison}. The development for Windows Phone 8 is supported by Visual Studio 2013 IDE.

\subsection{About C\# and .NET Framework}
C\# is a multi-paradigm programming language encompassing strong typing, imperative, declarative, functional, generic, object-oriented, and component-oriented programming disciplines \cite{cs1.0specs}. It was developed by Microsoft within its .NET initiative and its first version was released in 2002 \cite{cs1.0specs}. The latest release at the time of writing is C\# 5.0.
C\# was development at Microsoft by a team led by Anders Hejlsberg and is inspired by the C++ programming language, and was developed to overcome some of the flaws of other major programming languages.


\begin{table}[htbp]
  \centering
  \caption{Accessible Name for Various UI Elements}
    \label{tab:accessibleNameWP8}%
    \renewcommand{\arraystretch}{1.2}
    \begin{tabularx}{\textwidth}{l|X}
    \rowcolor{mygray}
    \textbf{Element type} & \textbf{Description} \\
    Static text & For TextBlock and RichTextBlock elements, an accessible name is automatically determined from the visible (inner) text. All of the text in that element is used as the name. \\ \hline
    Images & The XAML Image element does not have a direct analog to the HTML alt attribute of img and similar elements. To provide a name, AutomationProperties.Name can be used. \\ \hline
    Form elements & The accessible name for a form element should be the same as the label that is displayed for that element.\\ \hline
    Buttons and links & By default, the accessible name of a button or link is based on the visible text, using the same rules as described in the first row of the table. In cases where a button contains only an image, AutomationProperties.Name can be used to provide a text-only equivalent of the button's intended action. \\
    \end{tabularx}%
\end{table}%


