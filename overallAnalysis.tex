\chapter{Analysis}
Mobile devices and technologies play an important role in today's every day life. The number of mobile phones, tablets and other handheld devices has been increasing and according to a report by Cisco \cite{cisco}, the number of mobile-connected devices will exceed the number of people on earth by the end of 2014. The market is dominated by three main platforms: Android with market share of about 78\%, iPhone with 17\% and Windows Phone with 3\% \cite{phoneMkt}.

Since the thesis is developed with the Naviterier in mind, it is clear we will develop a client for one of the mobile platforms. UIProtocol clients for iPhone and Android are already implemented and we have therefore chosen the Windows Phone as the platform we will develop for. Reports show that Windows Phone is experiencing an overall growth in the world market\cite{phoneMkt} and we assume that Windows Phone will keep or strengthen its position.\\\\
Before moving onto the design and implementation chapters of the work, we will introduce the technology used for development, analysis of UIProtocol, overview of other navigation systems and accessibility analysis of the current mobile platforms.

\section{About Windows Phone 8}
The Windows Phone 8 is the first of Microsoft's mobile platform to use the Windows NT Kernel, which is the same kernel as the one in Windows 8 \cite{wp8kernel}. Therefore some parts of the API are the same for both systems. A significant subset of Windows Runtime is built into Windows Phone 8, with the functionality exposed to all supported languages \cite{wp8comparison}. This gives a developer the ability to use the same API for common tasks such as networking, working with sensors, processing location data and more. Therefore there is more potential for code reuse.

Also, Windows Phone 8 and windows 8 share the same .NET engine \cite{wp8comparison}. This is to deliver more stability and performance to the apps and improve battery life. Most new devices are now dual or quad-core, and the operating system and apps are expected to be faster thanks to this technology \cite{wp8comparison}. The development for Windows Phone 8 is supported by Visual Studio 2013 IDE.

\subsection{About C\# and .NET Framework}
C\# is a multi-paradigm programming language encompassing strong typing, imperative, declarative, functional, generic, object-oriented, and component-oriented programming disciplines \cite{cs1.0specs}. It was developed by Microsoft within its .NET initiative and its first version was released in 2002 \cite{cs1.0specs}. The latest release at the time of writing is C\# 5.0.
C\# was developed at Microsoft by a team led by Anders Hejlsberg and is inspired by the C++ programming language.

.NET Framework is a part of Windows OS which provides an virtual execution system called common language runtime (CLR) and also includes an extensive set of classes and libraries that offer a wide range of functionality \cite{csAndDotNet}. The specification called the Common Language Infrastructure (CLI) is an international standard by ISO/IEC 23270:2006 and ECMA-334 which specifies execution and development environment that allows multiple high-level languages to be used on different computer platforms without being rewritten for specific architectures \cite{csAndDotNet}. The CLR is Microsoft's implementation of the CLI standard. Other CLI implementations are Mono\footnotemark[1], DotGNU Portable.NET\footnotemark[2] and other.

In .NET, C\# source code is compiled into Common Intermediate Language and stored in an executable file, typically with exe or dll extensions \cite{csAndDotNet}. When executing the program, the CLR performs just-in-time compilation, producing executable machine-readable code and also handles garbage collection and other tasks. The key point is that the CIL code compiled from C\# conforms to the Common Type Specification (CTS) and therefore can interact with code that was generated from the .NET versions of Visual Basic, Visual C++, or any other CTS-compliant languages \cite{csAndDotNet}.

\footnotetext[1]{Mono Framework http://www.mono-project.com/Main\_Page}

\footnotetext[2]{DotGNU Portable.NET http://www.gnu.org/software/dotgnu/pnet.html}

