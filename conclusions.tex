\chapter{Conclusions and Future Work}
Navigation and mobility are key abilities to a comfortable living. Navigating outdoors is, thanks to global systems such as GPS significantly simplified. Indoor navigation remains a much less developed field. There are research projects that develop solutions for indoor navigation based on various technologies. One of such projects is NUIP from FEE CTU which combines NaviTerier (path planner and directions generator) with UIProtocol.

We analyzed the state of the art in indoor navigation and compared the accessibility features of the most common smartphone platforms. The reader was also introduced to UIProtocol, its architecture and how it functions.

The goal of the thesis was to implement an UIProtocol client supporting the basic features every UIP client needs to support. The platform of development was chosen to be Windows Phone 8. The client was expected to be able to be used within the NUIP project and consequently, we researched ways of making it more accessible to people with disabilities. We found, however, that the Windows Phone 8 platform is not suitable for visually impaired.

\section{Current Features}
Within the bachelor thesis we successfully implemented a basic UIProtocol client which implements:

\begin{description}
  \item[Basic UI controls] \hfill \\
  
  \item[Binding and model-wide binding] \hfill \\
  \item[Graceful degradation for unsupported elements] \hfill \\
  \item[Support for UI generator API] \hfill \\
  todo
  \item[Support for interpolated model updates (animations)] \hfill \\
  \item[Support for absolute and grid layouts] \hfill \\\\
  \item[Support for styling] \hfill \\\\
\end{description}

The features we implemented completely cover the functional requirements we have set in the Design chapter.

The non-functional requirements were also met. The UI loading times are below 0.5 s in all interfaces of the testing application. The execution profiling has shown increased CPU usage in some situations, but when idle, the consumption of phone's resources is normal.

\section{Future Work}
Implementing the full feature set of an UIProtocol client was not possible in the scope of this work. Therefore there is a number of features by which the client could be extended. In particular the extensions may include:

\begin{description}
  \item[StackPanel layout panel] \hfill \\
  The StackPanel is a simple layout panel that arranges its child elements into a single line that can be oriented horizontally or vertically. Currently, the app supports Grid and Canvas panels to which UI controls can be added. StackPanel should be implemented to widen the choice of layout panels. 
  \item[More UI controls] \hfill \\
  The client comes with a small set of implemented UI controls. New UI controls can be implemented, such as an audio element.
  \item[Wider support for styling] \hfill \\
  The app could implement more font styles, such as font family, font type (e.g. italic, bold).
  \item[Performance optimization] \hfill \\
  The performance of the current client is satisfactory, it could, however be improved by techniques such as interface pre-fetching.
  \item[Error handling] \hfill \\
  The client ignores errors received from the server, for example if a requested interface is not found at the server.
  \item[Support for different screen orientations] \hfill \\
When a user rotates the phone, the UI remains the same. this is not so important for the nuip project but an UIProtocol client should have this feature implemenred
refreshing only parts of UI that need to be refreshed.
\item[Screen reader friendliness] \hfill \\
Currently, the implementation redraws the whole UI. This approach was chosen for its simplicity. This approach, however, is not discouraged by the accessibility guidelines for WP8. todo toto bych mozna stihl predelat.
\end{description}



