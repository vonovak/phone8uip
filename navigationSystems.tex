\section{Navigation Systems Analysis}
\label{sec:nsa}
There is a number of research projects in the field of navigation systems for disabled. A large number of them are oriented toward the visually impaired or people with movement disabilities. Generally speaking, there are ongoing efforts to create maps for indoor environments, with the Google Indoor Maps\footnote[2]{Google Indoor Maps https://www.google.com/maps/about/partners/indoormaps/} being the head of this movement. Currently, the Google Indoor Maps are in beta and are not a priori intended for navigation but merely to provide the user with an approximate idea of where they are. In this chapter we will analyze some of the existing works which specifically address the problem of indoor navigation.

\subsection{NaviTerier}
NaviTerier \cite{naviterier} is a research project at FEE CTU which aims at the problem of navigating the visually impaired inside buildings. This system does not require any specialized technical equipment. It relies on a mobile phone with voice output which, for a visually impaired, is a natural way of communicating information. The navigation system works on a principle of sequential presentation of carefully prepared description of the building to the user by the mobile phone voice output. This system does not keep track of the user location. Instead, it breaks the directions into small pieces and then sequentially gives the pieces to the user who follows them and asks for next portion when ready.

This system is being integrated with UI Protocol platform, which is another research project of FEE CTU developed for the purpose of creating user interfaces customized to abilities and preferences of individual users. The result is navigation system called NaviTerier UIP (NUIP) \cite{balata} which combines the navigational part of NaviTerier and UIP - the context-sensitive UI generator that can adapt the user interface based on several criteria: device (navigation terminal, smartphone, etc.), user (visually impaired, elderly, ...) and environment (inside, outside, etc.). Over the course of development, the system has been tested several times with a total of about 100 visually impaired.


\subsection{PERCEPT}
A promising approach toward an indoor navigation system is shown in the PERCEPT \cite{percept} project. Its architecture consists of three system components: Environment, the PERCEPT glove and an Android client, and the PERCEPT server. In the environment there are passive (i.e. no power supply needed) RFID tags (R-tags) deployed at strategic locations in a defined height and accompanied with signage of high contrast letters and embossed Braille. The users' part of the environment are kiosks. Kiosks are where the user tells the system her destination. They are located at key locations of the building, such as elevators, entrances and exits and more. The R-tags are present here and the user has to find the one she needs and scan it using the glove.

The glove is used to scan the R-tags and also has buttons on it that the user can press to get instructions for the next part of the route, repeat previous instructions and get instructions back to the kiosk. Also, after scanning the R-tag the glove sends its information to the app running on user's Android phone. 

The Android app connects to the internet and downloads the directions from the PERCEPT server. These are then presented to the user through a text-to-speech engine and the user follows them. The system was tested with 24 visually impaired users of whom 85\% said that it provides independence and that they would use it.

\subsection{Blindshopping}
Another example of where RFID is used for indoor navigation is presented by Lopez et al. \cite{lopez}. The system is devised to allow visually impaired people to do shopping autonomously within a supermarket. The user is navigated by following paths marked by RFID labels on the floor. The white cane acts as an RFID reader and communicates with a smartphone which, as in other projects, uses TTS to give directions. The Android application is also used for product recognition using embossed QR codes placed on product shelves. The authors conducted a small usability study from which no conclusions can be drawn.

\subsection{System by Riehle et al.}
An indoor navigation system to support the visually impaired is presented in \cite{riehle}. The paper describes creation of a system that utilizes a commercial Ultra-Wideband (UWB) asset tracking system to support real-time location and navigation information. The paper claims that the advantage of using UWB is its resistance to narrowband interference and its robustness in complex indoor multipath environments. The system finds user position using triangulation and consists of four parts: tracking tag to be worn by the user, sensors that sense the position of the tracking tag, handheld navigator and a server which calculates the location of the tracking tag and communicates it to the navigator. The handheld device runs software which can produce audio directions to the user. In tests, the system proved useful; it was, however, tested only on blindfolded subjects. Testing showed that they found their way faster with the navigation system. 

\subsection{System by Treuillet at al.}
Treuillet and Royer \cite{sylvie} proposed a computer vision-based localization for blind pedestrian navigation assistance in both outdoors and indoors. The solution uses a real-time algorithm to match particular references extracted from pictures taken by a body-mounted camera which periodically takes pictures of the surroundings. The extraction uses 3D landmarks which the system first has to learn by going through a path along which the user later wants to navigate. It follows that the system is not suitable in environments that are visited for the first time. Accordingly to the authors, for the case when it has learned the way, the system performs well.

\subsection{System by Ozdenizci et al.}
Authors of \cite{indoorNFC} developed a system for general navigation and propose to use NFC tags. They claim the NFC navigation system is low cost and doesn't have the disadvantages present with the systems which use dead reckoning\footnote{Dead reckoning: The process of calculating one’s position by estimating the direction and distance travelled rather than by using landmarks or astronomical observations.} or triangulation\footnote{Triangulation is the process of determining the location of a point by measuring angles to it from known points at either end of a fixed baseline, rather than measuring distances to the point directly}. The proposed system consists of NFC tags spread in key locations of the building and a mobile device capable of reading the tags. The device runs an application which is connected to a server containing floor plans. The application is able to combine the information from the sensor with the floor plan and navigate the user using simple directions. The paper only proposes a system and does not contain any testing.

\subsection{System by Luis et al.}
Luis et al. \cite{luis} propose a system which uses an infrared transmitter attached to the white cane combined with Wiimote units (the device of the Wii game console). The units are placed so that they can determine the user's cane position using triangulation. The information from Wiimote units is communicated via Bluetooth to a computer which computes the position and then sends the directions to the user's smartphone via wifi. The distance limit of 10 m imposed by Bluetooth can be extended by using wifi but the authors do not consider this relevant because the proposal is in a preliminary evaluation phase. TTS engine running on the phone converts the directions to speech. The system has undergone preliminary testing with two blind and seven blindfolded users and the authors claim that "Results show that blind and blindfolded people improved their walking speed when the navigation system was used, which indicates the system was useful".

\subsection{Other}
There are also research works in the fields of robotics and artificial intelligence that study the problem of navigation. More specifically, they tackle the problem of real time indoor location recognition \cite{espinace}, \cite{quattoni}, \cite{bosch}. Some of these solutions allow for creating a reference map dynamically. Even though they proved to be useful in the domain of robotics and automotive industry, their applications to navigating people are limited, as they require expensive sensors and powerful computing resources. Wearing these devices would make the traveling of the users more difficult and limited. For these reasons, the solution proposed by Hesch and Roumeliotis \cite{hesch} is interesting because they integrated these devices (apart from the computing) into a white cane. However, the solution has the limitations of being too heavy and large.

\subsection{Conclusions and Comparison}
There are two main approaches to the problem of navigation. In the first, the navigation system consists of active parts which, using triangulation or other methods, are able to determine the user's position at all times and then give her directions based on knowing where she is.

In the second approach, the system does not possess the information about user's position at all times. Instead it synchronizes the position at the beginning of the navigation task and then gives the user directions broken into small chunks. When the user believes she reached the destination described by the first chunk, she asks for the next one and etc. The disadvantage of this approach is that the user can get lost and not end up at the expected location. This problem can be solved by adding more "synchronization points" to strategic locations of the building. These "synchronization points" are often done through NFC tags.

A quick review of the analyzed navigation systems is given in table \ref{tab:navSysComp}.
%sighted people

\begin{table}[htbp]
  \centering
  \caption{Quick comparison of the analyzed navigation systems}
  \label{tab:navSysComp}
 \renewcommand{\arraystretch}{1.2}
    \begin{tabularx}{\textwidth}{p{2.6cm}|X|X}
    \rowcolor{mygray}
    \textbf{System} & \textbf{active components} & \textbf{tested with} \\
    NaviTerier & mobile device, kiosk & visually impaired  \\ \hline
    PERCEPT & RFID, mobile device, kiosk & visually impaired  \\ \hline
    Lopez et al. & RFID, mobile device, kiosk & not stated \\ \hline
    Luis et al. & infrared, Wiimote & blindfolded users \\ \hline
    Riehle et al. & UWB tracking system & blindfolded users \\ \hline
    Treuillet at al. & camera & not stated \\ \hline
    Ozdenizci et al. & NFC, mobile device, kiosk & not stated \\
    \end{tabularx}%
\end{table}%



