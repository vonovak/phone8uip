%% FELthesis: LaTeX class for bachelor, master, and phd thesis in CTU FEL
%% template.tex: template file
%% (c) 2012 Vít Zýka, vit.zyka@seznam.cz
%%
%% 2012-12-17 v0.1 first version derived from cmpthesis.tex

\documentclass[bcl,draft]{felthesis} % or [...,czech] for thesis in Czech
%%\documentclass[msc,draft]{felthesis} % or [...,czech] for thesis in Czech
%%\documentclass[phd,draft]{felthesis} % or [...,czech] for thesis in Czech

%% --- your additional packages:
%\usepackage[utf8]{inputenc}
\usepackage[backend=biber]{biblatex}
\usepackage{booktabs}
\usepackage{multirow}
\usepackage{graphicx}

%% for code listings
\usepackage[final]{listings}
  \usepackage{courier}
 \lstset{
         basicstyle=\footnotesize\ttfamily, % Standardschrift
         %numbers=left,               % Ort der Zeilennummern
         numberstyle=\tiny,          % Stil der Zeilennummern
         %stepnumber=2,               % Abstand zwischen den Zeilennummern
         numbersep=5pt,              % Abstand der Nummern zum Text
         tabsize=2,                  % Groesse von Tabs
         extendedchars=true,         %
         breaklines=true,            % Zeilen werden Umgebrochen
         keywordstyle=\color{red},
    		frame=b,         
 %        keywordstyle=[1]\textbf,    % Stil der Keywords
 %        keywordstyle=[2]\textbf,    %
 %        keywordstyle=[3]\textbf,    %
 %        keywordstyle=[4]\textbf,   \sqrt{\sqrt{}} %
         stringstyle=\color{white}\ttfamily, % Farbe der String
         showspaces=false,           % Leerzeichen anzeigen ?
         showtabs=false,             % Tabs anzeigen ?
         xleftmargin=17pt,
         framexleftmargin=17pt,
         framexrightmargin=5pt,
         framexbottommargin=4pt,
         %backgroundcolor=\color{lightgray},
         showstringspaces=false      % Leerzeichen in Strings anzeigen ?        
 }
 \lstloadlanguages{% Check Dokumentation for further languages ...
         %[Visual]Basic
         %Pascal
         %C
         %C++
         %XML
         %HTML
         Java
 }
    %\DeclareCaptionFont{blue}{\color{blue}} 

  %\captionsetup[lstlisting]{singlelinecheck=false, labelfont={blue}, textfont={blue}}
  \usepackage{caption}
\DeclareCaptionFont{white}{\color{white}}
\DeclareCaptionFormat{listing}{\colorbox[cmyk]{0.43, 0.35, 0.35,0.01}{\parbox{\textwidth}{\hspace{15pt}#1#2#3}}}
\captionsetup[lstlisting]{format=listing,labelfont=white,textfont=white, singlelinecheck=false, margin=0pt, font={bf,footnotesize}}
%% end of for code listings


%% --- usefull draft packages
%%\usepackage[notref]{showkeys} % show labels for referencies
%%\usepackage{showlabels}       % similar
%%\usepackage{showidx}          % show index entries on every page

%% ======================================================== thesis info
\startThesisInfo
  \Title{Accessible UIP client for Windows Phone 8}
  \Author{Vojtech Novak}
  %\AuthorEmail{???@fel.cvut.cz} % optional
  %\ThesisUrl{http://fel.cvut.cz/???/???-bsc.pdf} % optional
  \Date{\today}
  \Department{\\Dept. of Computer Graphics and Interaction}
  \Advisor{Macik Miroslav Ing.}
  \KeywordsCz{Navigace, Generování UI, Přístupnost, Windows Phone 8, UIP}
  \KeywordsEn{Navigation, UI Generation, Accessibility, Windows Phone 8, UIP}
  %\AssignmentPage{assignment.pdf} % insert official assignment if given
\stopThesisInfo

%% ============================== your definitions (abbreviations etc.)
%%\def\Ax{\mathbf{A}_{x}}

%% =========================================================== settings
\addbibresource{biblatex.bib} % bibliography file

\graphicspath{{logo}{fig}} % subdirectories where TeX finds pictures

%% ========================================================== text body
\begin{document}

\MakeTitle

\startFrontMatter
  \startAcknowledgement
Text of acknowledgement\dots
\stopAcknowledgement

\endinput
%%
%% End of file `acknowledgement.tex'.

  \startDeclaration
\ifCzech
  Prohlašuji, že jsem předloženou práci vypracoval samostatně,
  a~že jsem uvedl veškeré použité informační zdroje v~souladu
  s~Metodickým pokynem o~dodržování etických principů při přípravě
  vysokoškolských závěrečných prací.
\fi
\ifEnglish
  I declare that I worked out the presented thesis independently
  and I quoted all used sources of information in accord with
  Methodical instructions about ethical principles for writing an
  academic thesis.
\fi
\stopDeclaration

\endinput
%%
%% End of file `declaration.tex'.

  \startAbstractCz
  Text abstraktu česky\dots
\stopAbstractCz

\startAbstractEn
  This work describes the development of an accessible UIP client application for Windows Phone 8.
\stopAbstractEn

\endinput
%%
%% End of file `abstract.tex'.

  \TableOfContents
  \startAbbreviations{%
  The list of abbreviations used in this document
}
\abbrv[FEE CTU] Faculty of Electrical Engineering of the Czech Technical University in Prague
\abbrv[UIP]  UI Protocol developed for research purposes at the FEE CTU
\abbrv[TTS]  Text-to-Speech
\abbrv[OS]  Operating System
\abbrv[API]  Application Programming Interface

\stopAbbreviations

\endinput
%%
%% End of file `abbreviations.tex'.

\stopFrontMatter

\startBodyMatter
  %\includeonly{ch01}
  \chapter{UIProtocol}
This chapter introduces the reader to the UIProtocol, its architecture and communication between client and server.
%Analyse UIP platform, focus on UIP client design.

\section{About UIProtocol}
Universal Interface Protocol (UIProtocol) is a user interface specification language \cite{uip} being developed at FEE CTU for research purposes. At the time of writing this thesis, the specification is not publicly available. UIProtocol provides means for describing user interfaces and transferring data related to interaction between user and an UIProtocol based application. It is designed to be cross-platform, programming language independent and easily localized.\\
UIProtocol is an XML based application protocol that allows for describing the hierarchical structure of the GUI along with the placement and visual appearance of the containers and components. It is designed for a client-server system and for facilitating client-server applications it defines the communication rules between client and server. The communication  is based on exchange of XML documents which contain all components and values needed for rendering the UI. The client first initiates the communication and receives respective XML description from the server. The description can be of four different types: interfaces, i.e. the UI components and containers, models which contain the data displayed and actions. The communication from client to server only consists of event descriptions. For example, when a user presses a button, the event information is sent to the server which responds by model and/or interface update.\\
An example of such situation may be a user requesting a weather app to his handheld device. As he enters his location and presses a button to request the weather information, part of the job is done directly by the client and the other is sent to the server. The part done directly at the client are easy tasks, such as visual effects when pressing the button. Even these effects can be, ahead of time, specified by information from the server. The request for weather is then sent to the server (event), which processes the information and responds by sending the interface structure, components and the weather information. This is then displayed to the user who than has other options to interact with the app.

The documents of UIProtocol can be sent in either direction usually through a single channel without waiting for a request, e.g. server can send updates to the client as soon as the displayed information needs to be updated, without having to wait for an update request. Should an application not communicate with a remote server, there is the possibility of both client and server running on the same machine although this is not a typical usage.

\subsection{UIProtocol Client}
UIProtocol client is thin, i.e. no application code is executed on the client side. The device running the client is thought to be the the one user directly uses, that is, it renders the content to the user and receives input from them. From the UIProtocol point of view, the client device is also considered insecure, i.e. the device may be misused to send invalid data to the server and may be used to attack it. The UIP client may not implement the whole feature set defined by UIProtocol. What has to be implemented is the minimal functionality that is able to render a user interface, sending event information to the server and update the application it by data coming from it.

\subsection{UIProtocol Server}
UIProtocol is the part of the architecture which is responsible for evaluating the client events and sending a correct response - this is where the application logic is executed. It must be able to service multiple client simultaneously and is intended to run on a machine which is considered safe.

\subsection{Elements of UIProtocol Communication}
As mentioned previously, the information exchange between client and server concerns Interfaces, Models, Actions (which are sent from server to client) and events (sent from client to server). In the following subsections we will describe these in greater detail and also include more information on UIP syntax.

\subsubsection{Syntax of UIProtocol}
The UIP document syntax is shown below in Listing \ref{uipSyntax} which shows the four possible tags with the root element, with the actions tag being optional. The tags define the behavior of the application and are covered later in the chapter, with the exception of actions tag.

\lstinputlisting[label=uipSyntax,caption=UIP document Syntax]{sources/uipSyntax.xml}

Every UIProtocol document must contain an XML header with the version and encoding (UTF-8 recommended).

\subsubsection{Interfaces}
Interface describes the structure and components of the user interface. Every interface can nest containers and elements that form a part of user interface. An example can be seen in Listing \ref{uipInterface}. The listing includes containers and elements of different types, for example "public.input.text" is a standard component which will be rendered as an element into which a user can enter text. It also shows how interfaces can be embedded. This is done by including an element or container with class name corresponding to different interface's class. The interfaces are uniquely identified by the class attribute (unlike most other objects in UIProtocol and other markup languages identified by id attribute). Note that element tag can have an id attribute, as shown.\\

\lstinputlisting[label=uipInterface,caption=Interface Description Example]{sources/interfaces.xml}

\subsubsection{Events}
Events inform the server that some action was triggered at the client side (e.g. a button press) or that there was some other update (e.g. change in sensor readings). This is the basic mechanism of client to server communication and therefore has to be supported by client. The event element contains a unique id which specifies the event source. An event can contain any number of properties which describe it more closely. An example is shown in Listing \ref{uipEvents}.
\lstinputlisting[label=uipEvents,caption=Events Example]{sources/uipEvents.xml}

\subsubsection{Models}
Models serve as the storage for data that is displayed to the user. A model contains data in properties which are uniquely identified by key attribute. When changing the contents of a model, all of the UI elements bound to the model by the key attribute are updated. UIP provides support for this binding and therefore updating the app contents is made easy. For example, there may be two representations of humidity level in a given environment (text and a graphical representation). If they are both bound to the same model, changing the respective property within that model will be immediately reflected.

\subsubsection{Properties}
Properties are the most nested elements of an UIP document and define the visual appearance, positioning of GUI objects and more. Every property has to have a name, which defines what feature of the connected object is described by this property. For example the property in Listing \ref{uipInterface} with name text defines the text displayed in the text field. Value is a constant that will define the text. The mentioned property also contains the key attribute which binds the displayed value to a model. The part before the colon references a model and the part after colon references a property within the model. That is, if there is a model gui with property fstName, the value of the property is used. Moreover, whenever the value of the referenced property is updated, the update is automatically reflected in all other bound properties.

\endinput
  \section{Accessibility of Current Mobile Platforms}
In this chapter we will analyze the accessibility features of today's most common mobile platforms. Since this thesis is about development of a UIP client for Windows Phone 8, we will put emphasis on this OS.

\subsection{Windows Phone 8 Accessibility}
This section covers all of the features for ease of access that are included in the Windows Phone 8 operating system. For the purposes of this project, we are particularly interested in features that may help disabled users. From this point of view, one of the most important are the voice commands and speech recognition features which Windows phone 8 has built-in and which support a wide range of languages.

\subsubsection{Speech Features}
Users can interact with the phone using speech. There are three speech components that a developer can integrate in her app and the user can take advantage of them: voice commands, speech recognition, and text-to-speech (TTS). We will explore these features in the following paragraphs.
At the time of writing, the speech features support 15 major languages ranging from English to Russian or even English with the Indian accent. Czech, however, is not supported. To use the speech features the user has to download a language pack.

\paragraph{Speech Recognition}
Users can give input to an app or accomplish tasks with it using speech recognition. An example usage can be dictating content of an SMS \cite{phone8speech}. This is very similar to the Voice Command feature, but the key difference is that speech recognition occurs when user is in the app, and Voice Commands occur from outside of the app \cite{phone8speech}. The second key difference is that the Voice Commands are defined on a finite and usually small set of words (commands), whereas the Speech Recognition should recognize words from a much larger dictionary – in ideal case a whole human language.

\paragraph{Voice Commands}
When a user installs an app, they can automatically use voice to access it by speaking "open" or "start", followed by the app name \cite{wp8voice}. The range of actions that can be triggered by Voice Commands is much wider, the full list of available speech commands that are provided by the operating system is listed in table \ref{tab:w8sc}.\\
A developer can also define her own set of voice commands and allow users not only to open the app using voice but also to carry out more advanced tasks within the app \cite{wp8voice}. This is important for our work since it allows for exposing a wider range of commands to potential visually impaired users. Note that technically, this still happens from the outside of the app, as described in the previous section.

\paragraph{Text to Speech (TTS)}
TTS can be used to speak text to the user via the phone's speaker or headset. The spoken text can be simple strings or strings formatted according to the industry-standard Speech Synthesis Markup Language (SSML) Version 1.0 \cite{phone8speech}. TTS is also used in some of the other features for ease of access which are covered in the next section.

\paragraph{Other Speech Features}
A feature named Speech for phone accessibility allows the following \cite{wp8voice}:
\begin{enumerate}
\item Talking caller ID\\
When getting a call or receiving a text, the phone can announce the name of the caller or the number.
\item Speech-controlled speed dial\\
User can assign a number to a person from the contact list and then say Say "Call speed dial number" (where number is the assigned number) to call the person. Assigning the speed dial number is also speech-enabled.
\item Read aloud incoming text messages
\end{enumerate}

\subsubsection{Other Tools for Ease of Access}
Windows phone 8 comes with more features for ease of access which can help lightly visually impaired users. User can change font size in selected build-in apps (The API for determining if the font size was changed by user is available only from WP 8.1 and the app developer can decide whether she will respect the user font size settings. \cite{wp8accText}), switch the display theme to high-contrast colors and use the screen magnifier \cite{wp8screenreader}. Mobile Accessibility is a set of accessible apps with a screen reader, which helps use the phone by reading the application content aloud. These applications include phone, text, email, and web browsing \cite{wp8screenreader}. When Mobile Accessibility is turned on, notifications like alarms, calendar events, and low battery warnings will be read aloud. This feature, however, is only available in version 8.0.10501.127 \cite{wp8screenreader} or later. For an unknown reason, an update to this version is not available for our device.

Windows Phone 8.1 contains a new accessibility feature, called Narrator, which allows to read aloud the content of the screen \cite{wp8narrator}. The feature is intended for visually impaired but we do not have enough information to evaluate its usability.

\subsubsection{Conclusions}
Windows Phone 8 platform offers some features to make its accessibility to disabled users easier. However, there are still gaps to be filled such as the non-existence of a built-in screen reader. Its absence puts both Windows Phone 8 and 8.1 usage out of the question for visually impaired users. There is an limited screen reader option available from version 8.0.10501.127 \cite{wp8screenreader} but this only works with some apps and update to this version is not available to all devices at the time of writing. The platform has recently been experiencing growth of about 6\% in some countries of Europe but only slow growth others \cite{phone8market} and it cannot be estimated how much effort will be put into the development of more accessibility features. It should be noted that the other two major platforms, iOS and Android both include a screen reader.





\subsection{Android Accessibility}
Similarly to the previous section, here we will analyze the accessibility options for devices running the Android operating system. We will analyze the features of the latest Android OS released at the time of writing, which is version 4.4, code name KitKat. It should be noted that there were no major updates to the accessibility options since Android 4.2.2 Jelly Bean.

\subsubsection{Speech Features}
Android too offers the option to interact with the device using speech and has some interesting accessibility features. Compared to Windows Phone 8, Android offers a wider language support. Similarly to Windows Phone 8, an Android developer can take advantage of speech recognition and text-to-speech (TTS). Android comes with a number of built-in voice commands but unlike the Windows Phone, Android does not allow developers to expose their own voice commands. The last important feature on Android is TalkBack screen reader.
At the time of writing, the speech recognition supports more than 40 languages including even minor languages such as Czech. The text to speech does not have such a wide support.

\paragraph{Speech Recognition}
Users can give input to an app or accomplish tasks with it using speech recognition. An example usage can be dictating content of an SMS. As mentioned before, this feature supports many languages but on the other hand, internet connection is required \cite{androidRecog} because the recognition is done at Google servers. We do not consider this a major drawback, as nowadays a mobile internet connection is more available than ever.

\paragraph{Voice Action Commands}
In Android, Voice Action Commands are closely related to the Google Now feature. Google Now has a wide range of uses. It can also serve well to disabled people because it allows to get information using voice.
In general, Google Now should provide the user with relevant information when they need it. Google describes it by the phrase “The right information at just the right time”. This includes telling the user the weather forecast, showing the best route to work, calling someone, creating a reminder and more \cite{googleNow}. The full list of Voice action Commands is in Table \ref{tab:asc}.

Note that for some commands, the system gives you a spoken answer. The current drawback of the system is that it only supports English, French, German, Spanish, and Italian \cite{androidTTS}. With other languages, user can only make a voice-induced Google search with no voice response.

\paragraph{Text to Speech (TTS)}
TTS can be used to speak text to the user via the phone's speaker or headset. The spoken text can be simple strings. The industry-standard Speech Synthesis Markup Language (SSML) is not mentioned in the API documentation. Supported are only major world languages, enlisted in previous section. TTS is also used in TalkBack which is described in the next section.

\paragraph{Other Speech Features}
TalkBack is an important functionality that strives for more accessible phone control for visually impaired \cite{androidAccbility}. Basically, it is a touch-controlled screen reader. When enabled, user can drag finger across the screen selecting the components and getting their acoustic description. By double tapping anywhere in the screen, user can open/use the last selected item. TalkBack also supports gestures. This way, a user can get a complete description of the user interface \cite{androidAccbility}. The blog post of a blind accessibility engineer from Mozilla Foundation \cite{mozillaguy} claims that visually impaired users of this system still have to overcome some obstacles.

\subsubsection{Other Tools for Ease of Access}
Android too comes with more features for ease of access which can help lightly visually impaired users which include change of font size and the screen magnifier.

\subsubsection{Conclusions}
Compared to Windows Phone, Android has better accessibility options. It includes the usual functions, such as text to speech, speech recognition or font size settings. It also offers a built-in screen reader, called TalkBack. Android aims to be usable even for visually impaired.





\subsection{iOS Accessibility}
In this chapter, we will cover the accessibility of Apple's iOS. Again we consider the latest iOS released, which is version 7.0.4. Overall, the accessibility features of iOS are very similar to those of Android and therefore we will describe the features more briefly.

\subsubsection{Speech Features}
As with the previous two platforms, iOS also offers users to interact with a device using speech. iOS supports speech recognition and text-to-speech in 15 major languages (the same number as Windows Phone 8).
iOS also comes with a number of built-in voice commands \cite{iosAccbility} but does not allow developers to expose their own voice commands.

\paragraph{Speech Recognition - Dictation}
Users can give input to an app or accomplish tasks with it using speech recognition. An example usage can be dictating content of a text. As mentioned before, this feature supports 15 languages and requires an internet connection.

\paragraph{Voice Control - Siri}
Siri in iOS can be thought of as an equivalent to Android's Google Now. Siri can send emails, set reminders and more \cite{iosAccbility}. If asked a question, it can read aloud the answer. 

\paragraph{Text to Speech (TTS)}
TTS can be used to speak text to the user via the phone's speaker or headset and this feature was added only recently, in iOS 7.0. The spoken text can be simple strings. The industry-standard Speech Synthesis Markup Language (SSML) is not mentioned in the API documentation.

\paragraph{Other Speech Features}
It could be said that Google's TalkBack is Apple's VoiceOver. Both offer very similar functions and they allow reading the content of the screen based on touch input and control of the device by gestures \cite{iosVoiceOver}.
The mentioned blog post of the blind accessibility engineer from Mozilla Foundation favors VoiceOver over TalkBack \cite{mozillaguy}.

\subsubsection{Other Tools for Ease of Access}
iOS too comes with more features for ease of access which can help lightly visually impaired users. The user can change font size, invert Colors and use the screen magnifier (Zoom) \cite{iosAccbility}.
iOS devices also support a number of Bluetooth wireless braille displays out of the box \cite{iosAccbility}. User can pair their braille display with the device and start using it to navigate it with VoiceOver. iPad, iPhone, and iPod touch include braille tables for more than 25 languages.

\subsubsection{Conclusions}
Apple's iOS was the first to offer advanced accessibility features and the first to become usable for visually impaired users. The accessibility options are comparable with the  feature set of Android. iOS allows developers to create apps accessible for a wide range of users.

\subsection{Comparison of Analyzed Platforms}
A quick overview on the accessibility features of the three most common mobile platforms that we analyzed is given in table \ref{tab:accbilityComparison}.

\begin{table}[htbp]
  \centering
  \caption{Quick comparison of the accessibility features of today's mobile platforms}
  \label{tab:accbilityComparison}
 \renewcommand{\arraystretch}{1.2}
    \begin{tabularx}{\textwidth}{X|X|X|X|X}
    \rowcolor{mygray}
    \textbf{Platform} & \textbf{Built-in screen reader} & \textbf{Text to speech} & \textbf{Speech recognition} & \textbf{Built-in braille display support}\\
    Windows Phone 8 & no\footnotemark[1] & yes & yes & no \\ \hline
    Android & yes & yes & yes & no\\ \hline
    iPhone & yes & yes & yes & yes \\
    \end{tabularx}%
\end{table}%
\footnotetext[1]{Windows Phone 8.1 contains a screen reader feature called Narrator}




  \subsection{Android Accessibility}
Similarly to the previous section, here we will analyze the accessibility options for devices running the Android operating system. We will analyze the features of the latest Android OS released at the time of writing, which is version 4.4, code name KitKat. It should be noted that there were no major updates to the accessibility options since Android 4.2.2 Jelly Bean.

\subsubsection{Speech Features}
Android too offers the option to interact with the device using speech and has some interesting accessibility features. Compared to Windows Phone 8, Android offers a wider language support. Similarly to Windows Phone 8, an Android developer can take advantage of speech recognition and text-to-speech (TTS). Android comes with a number of built-in voice commands but unlike the Windows Phone, Android does not allow developers to expose their own voice commands. The last important feature on Android is TalkBack screen reader.
At the time of writing, the speech recognition supports more than 40 languages including even minor languages such as Czech. The text to speech does not have such a wide support.

\paragraph{Speech Recognition}
Users can give input to an app or accomplish tasks with it using speech recognition. An example usage can be dictating content of an SMS. As mentioned before, this feature supports many languages but on the other hand, internet connection is required \cite{androidRecog} because the recognition is done at Google servers. We do not consider this a major drawback, as nowadays a mobile internet connection is more available than ever.

\paragraph{Voice Action Commands}
In Android, Voice Action Commands are closely related to the Google Now feature. Google Now has a wide range of uses. It can also serve well to disabled people because it allows to get information using voice.
In general, Google Now should provide the user with relevant information when they need it. Google describes it by the phrase “The right information at just the right time”. This includes telling the user the weather forecast, showing the best route to work, calling someone, creating a reminder and more \cite{googleNow}. The full list of Voice action Commands is in Table \ref{tab:asc}.

Note that for some commands, the system gives you a spoken answer. The current drawback of the system is that it only supports English, French, German, Spanish, and Italian \cite{androidTTS}. With other languages, user can only make a voice-induced Google search with no voice response.

\paragraph{Text to Speech (TTS)}
TTS can be used to speak text to the user via the phone's speaker or headset. The spoken text can be simple strings. The industry-standard Speech Synthesis Markup Language (SSML) is not mentioned in the API documentation. Supported are only major world languages, enlisted in previous section. TTS is also used in TalkBack which is described in the next section.

\paragraph{Other Speech Features}
TalkBack is an important functionality that strives for more accessible phone control for visually impaired \cite{androidAccbility}. Basically, it is a touch-controlled screen reader. When enabled, user can drag finger across the screen selecting the components and getting their acoustic description. By double tapping anywhere in the screen, user can open/use the last selected item. TalkBack also supports gestures. This way, a user can get a complete description of the user interface \cite{androidAccbility}. The blog post of a blind accessibility engineer from Mozilla Foundation \cite{mozillaguy} claims that visually impaired users of this system still have to overcome some obstacles.

\subsubsection{Other Tools for Ease of Access}
Android too comes with more features for ease of access which can help lightly visually impaired users which include change of font size and the screen magnifier.

\subsubsection{Conclusions}
Compared to Windows Phone, Android has better accessibility options. It includes the usual functions, such as text to speech, speech recognition or font size settings. It also offers a built-in screen reader, called TalkBack. Android aims to be usable even for visually impaired.


  \section{iOS Accessibility}
In this chapter, we will cover the accessibility of Apple's iOS. Again we consider the latest iOS released, which is version 7.0.4. Overall, the accessibility features of iOS are very similar to those of Android and therefore I will describe the features more briefly.
\subsection{Speech Features}
As with the previous two platforms, iOS also offers users o interact with a device using speech. Apple was the first one to introduce the features for people with disabilities, such as VoiceOver. iOS supports speech recognition and text-to-speech in 15 major languages (the same number as Windows Phone 8).
iOs also comes with a number of built-in voice commands but does not allow developers to expose their own voice commands.
\subsubsection{Speech Recognition - Dictation}
Users can give input to an app or accomplish tasks with it using speech recognition. An example usage can be dictating content of a text. As mentioned before, this feature supports 15 languages and requires an internet connection.
\subsubsection{Voice Control - Siri}
Siri in iOS can be thought of as an equivalent to Android's Google Now. Siri can send emails, set reminders and more. If asked a question, it can read aloud the answer. 
\subsubsection{Text to Speech (TTS)}
TTS can be used to speak text to the user via the phone's speaker or headset and this feature was added only recently, in iOS 7.0. The spoken text can be simple strings. The industry-standard Speech Synthesis Markup Language (SSML) is not mentioned in the API documentation.
\subsubsection{Other Speech Features}
It could be said that Google's TalkBack is Apple's VoiceOver. Both offer very similar functions and the reason for existence is reading the content of the screen based on touch input and control of the device by gestures.
The mentioned blog post of the blind accessibility engineer from Mozilla Foundation favors VoiceOver over TalkBack \cite{mozillaguy}.

\subsection{Other Tools for Ease of Access}
iOS too comes with more features for ease of access which can help lightly visually impaired users. The user can change font size, invert Colors and use the screen magnifier (Zoom).
iOS devices also support more than 40 Bluetooth wireless braille displays out of the box. User can pair their braille display with the device and start using it to navigate it with VoiceOver. In addition, iPad, iPhone, and iPod touch include braille tables for more than 25 languages.

\subsection{Conclusions}
Apple's iOS was the first to offer advanced accessibility features and the first to become usable for visually impaired users. The accessibility features are comparable with the accessibility feature set of Android. iOS allows developers to create apps acesible for a wide range of users.
\endinput
  \section{Navigation Systems Analysis}
\label{sec:nsa}
There is a number of interesting papers in the field of navigation systems for disabled. A large number of them is oriented toward visually impaired or people with movement disabilities. Generally speaking, there are ongoing efforts to create maps for indoor environments, with the Google Indoor Maps being the head of this movement. Currently, the Google Indoor Maps are in beta and are not a priori intended for navigation but merely to provide the user with an approximate idea of where they are. In this chapter we will analyze some of the existing works which specifically address the problem of indoor navigation.

\subsection{NaviTerier}
NaviTerier \cite{naviterier} is a research project at FEE CTU which aims at the problem of navigating visually impaired inside buildings. This system does not require any specialized technical equipment. It relies only on mobile phones with voice output, which is a non-problematic way of communicating information. Instead of introducing a very new technology, it employs context awareness to provide adaptations to user-specific abilities. The navigation system works on a principle of sequential presentation of carefully prepared description of the building to the user by the current mobile phone voice output. This system does not keep track of the user location. Instead, it breaks the directions into smaller pieces and then sequentially gives the pieces to the user who follows them and asks for next portion when ready. This system was tested with visually impaired users.\\
This system was combined with UI Protocol platform, which is another research project of FEE CTU developed for the purpose of creating user interfaces customized to abilities and preferences of individual users. The result is navigation system called NaviTerier UIP (NUIP) \cite{balata} which combines the navigational part of NaviTerier and UIP - the context-sensitive UI generator that adapts the user interface according to navigation terminals and personal devices used during the navigation.

\subsection{PERCEPT}
A promising approach is shown in the PERCEPT \cite{percept} project. Its architecture consists of the three system components: the Environment, the PERCEPT glove and Android client, and
the PERCEPT server. In the environment there are passive (i.e. no power supply needed) RFID tags (R-tags) deployed at strategic locations in a defined height and accompanied with signage of high contrast letters and embossed Braille. The next part of the environment are the Kiosks. Kiosks are where the user tells the system her destination. They are located at key locations of the building, such as elevators, entrances and exits and more. The R-tags are present here and the user has to find the one she needs and scan it using the glove.
The glove is used to scan the R-tags and also has buttons on it that the user can press to get the instructions for the part of the route, repeat previous instructions and get instructions back to the kiosk. Also, after scanning the R-tag the gloves sends its information to the app running on user's Android phone. The app connects to the internet and downloads the directions from the PERCEPT server. These are then presented to the user through a text-to-speech engine and the user follows them. The system was tested with 24 visually impaired users.

\subsection{System by Lopez et al.}
Another example of RFID use is presented in Lopez et al. \cite{lopez} where user is navigated by following paths marked by RFID labels on the floor. The white cane acts as an RFID reader and communicates with a smartphone which, as in other projects, uses TTS to give directions.

\subsection{System by Luis et al.}
Luis et al.\cite{luis} propose a system which uses an infrared transmitter attached to the white cane combined with Wiimote units (the device of the Wii game console) placed so that they can determine the user's cane position using triangulation. The information from Wiimote units is communicated via Bluetooth to a computer which computes the position and then sends the directions to the user's smartphone via wifi. TTS engine running on the phone converts the directions to speech. The system has undergone preliminary testing with five blindfolded users.

\subsection{System by Riehle et al.}
An indoor navigation system to support the visually impaired is presented in \cite{riehle}. The paper describes creation of a system that utilizes a commercial Ultra-Wideband (UWB) asset tracking system to support real-time location and navigation information. The paper claims that the advantage of using UWB is its resistance to narrowband interference and its robustness in complex indoor multipath environments. The system finds user position using triangulation and consists of four parts: tracking tag to be worn by the user, sensors that sense the position of the tracking tag, handheld navigator and a server which calculates the location of the tracking tag and communicates it to the navigator. The handheld device runs software which can produce audio directions to the user. In tests, the system proved useful; it was, however, tested only on blindfolded people.

\subsection{System by Treuillet at al.}
Treuillet and Royer \cite{sylvie} proposed a vision-based localization for blind pedestrian navigation assistance in both outdoors and indoors. The solution uses a real-time algorithm to match particular references extracted from pictures taken by a body-mounted camera which periodically takes pictures of the surroundings. The extraction uses 3D landmarks which the system first has to learn by going through a path along which the user later want to navigate. It follows that the system is not suitable in environments that are visited for the first time. For the case when it has learned the way, the system performs well.

\subsection{System by Ozdenizci et al.}
Authors of \cite{indoorNFC} developed a system for general navigation and propose to use NFC tags. They claim the NFC navigation system is low cost and doesn't have the disadvantages present with the systems which use dead reckoning or triangulation. The proposed system consists of NFC tags spread in key locations of the building and a mobile device capable of reading the tags. The device runs an application which is connected to a server containing floor plans. The application is able to combine the information from the sensor with the floor plan and able to navigate the user using simple directions.

\subsection{Other}
There are also research works in the fields of robotics and artificial intelligence that study the problem of navigation. More specifically, they tackle the problem of real time indoor scenes recognition \cite{espinace}, \cite{quattoni}, \cite{bosch}. Some of these solutions allow for creating the reference map dynamically. Even though they proved to be useful in the domain of robotics and automotive industry, their applications to navigating people are limited, as they require expensive sensors and powerful computing resources. Wearing these devices would make the traveling of the users more difficult and limited. For these reasons, the solution proposed by Hesch and Roumeliotis \cite{hesch} is interesting because they integrated these devices (apart from the computing) into a white cane. However, the solution has the limitations in being too heavy and large, and the laser scanner being directional.

\subsection{Conclusions and Comparison}
There are two main approaches to the problem of navigation. In the first, the navigation system consists of active parts which, using triangulation or other methods, are able to determine the user's position at all times and then give her directions based on knowing where she is.\\In the second approach, the system does not possess the information about user's position at all times. Instead it synchronizes the position at the beginning of the navigation task and then gives the user directions broken into small chunks. When the user believes she reached the destination described by the first chunk, she asks for the next one and etc. The disadvantage of this approach is that the user can get lost and not end up at the expected location. This problem can be solved by adding more "synchronization points" to strategic locations of the building. These "synchronization points" are often done through NFC tags. A quick review of the analyzed navigation systems is given in table \ref{tab:navSysComp}.
%sighted people

\begin{table}[htbp]
  \centering
  \caption{Quick comparison of the analyzed navigation systems}
  \label{tab:navSysComp}
 \renewcommand{\arraystretch}{1.2}
    \begin{tabularx}{\textwidth}{p{2.6cm}|X|X}
    \rowcolor{mygray}
    \textbf{System} & \textbf{Components} & \textbf{Tested with} \\
    NaviTerier & mobile device, kiosk & visually impaired  \\ \hline
    PERCEPT & RFID, mobile device, kiosk & visually impaired  \\ \hline
    Blindshopping & RFID, mobile device & visually impaired \\ \hline
    Luis et al. & infrared, Wiimote, mobile device & blindfolded users \\ \hline
    Riehle et al. & UWB tracking system, mobile device & blindfolded users \\ \hline
    Treuillet at al. & camera, laptop & not found \\ \hline
    Ozdenizci et al. & NFC, mobile device & not relevant \\
    \end{tabularx}%
\end{table}%




  \chapter{Appendix}

% Table generated by Excel2LaTeX from sheet 'List1'
\begin{table}[htbp]
  \centering
  \caption{Windows Phone 8 Speech Commands}
    \begin{tabular}{ll}
    \toprule
    To    & Say this \\
    \midrule
    \multirow{3}[2]{*}{Call someone from your contact list} & "Call contact name" \\ 
          & (where contact name is the name of someone in your contact list) \\
          & If the person has only one phone number in your contact card, the call will start. If he or she has multiple phone numbers, you'll see an option to choose one. \\
    \multirow{2}[2]{*}{Call any phone number} & "Call phone number" \\
          & (where phone number is any phone number, whether it belongs to a contact or not) \\
    Redial the last number & "Redial" \\
    \multirow{3}[2]{*}{Send a text message} & "Text contact name" \\
          & (where contact name is the name of someone in your contact list) \\
          & This will start a text message to that person. Then you can dictate and send the message—hands-free. \\
          &  \\
    Call your voicemail & "Call voicemail" \\
    \multirow{2}[2]{*}{Open an application} & "Open application" or "Start application" \\
          & (where application is the name of any application on your phone, such as "Calendar," "Maps," or "Music") \\
    \multirow{3}[2]{*}{Search the web} & "Find search term" or "Search for search term" \\
          & (where search term is what you're looking for) \\
          & If you say "Find local pizza," for example, Bing will bring up a map of nearby pizza places. \\
    \bottomrule
    \end{tabular}%
  \label{tab:addlabel}%
\end{table}%

% Table generated by Excel2LaTeX from sheet 'List1'
\begin{table}[htbp]
  \centering
  \caption{Android Speech Commands}
    \begin{tabular}{lll}
    \toprule
    \textbf{Say} & \textbf{Followed by} & \textbf{Examples} \\
    \midrule
    "Open" & App name & "Open Gmail" \\
    "What does my day look like tomorrow?"or "Show me my schedule for the weekend." &       & Say "What does my day look like tomorrow?" to see tomorrow's agenda. \\
    "Create a calendar event" & "Event description" \& "day/date" \& "time" & "Create a calendar event: Dinner in San Francisco, Saturday at 7:00PM" \\
    "Listen to TV" & Displays TV cards relevant to the TV show that's currently being broadcast & While a TV show is being broadcast, say "Listen to TV" \\
    "Map of" & Address, name, business name, type of business, or other location & "Map of Golden Gate Park, San Francisco." \\
    "Directions to" or & \multirow{2}[2]{*}{Address, name, business name, type of business, or other destination} & "Directions to 1299 Colusa Avenue, Berkeley, California" or \\
    "Navigate to" &       & "Navigate to Union Square, San Francisco." \\
    "Post to Google+" & What you want posted to Google+ & "Post to Google+ I'm going out of town." \\
    "What's this song?" &       & When you hear a song, ask "What's this song?" \\
    "Remind me to" & What you want to be reminded about, and when or where & "Remind me to call John at 6PM." \\
    "Go to" & Search string or URL & "Go to Google.com" \\
    \multirow{3}[2]{*}{"Send email"} & "To" \& contact name, & \multirow{3}[2]{*}{"Send email to Hugh Briss, subject, new shoes, message, I can’t wait to show you my new shoes, period."} \\
          & "Subject" \& subject text, &  \\
          & "Message" \& message text (speak punctuation) &  \\
    "Note to self" & Message text & "Note to self: remember the milk" \\
    "Set alarm" & "Time" or "for" \& time, such as "10:45 a.m." or "20 minutes from now," "Label" \& name of alarm & "Set alarm for 7:45 p.m., label, switch the laundry" \\
    "Listen to" & Play music in the Google Play Music app by speaking the name of a song, artist, or album & "Listen to: Smells Like Teen Spirit" \\
    "Call" & The name of one of your contacts & "Call George Smith" \\
    \bottomrule
    \end{tabular}%
  \label{tab:addlabel}%
\end{table}%

  %Lorem ipsum dolor sit amet \cite{kastenholz}.
  %\chapter{Android Accessibility}
Similarly to the previous chapter, here we will analyze the accessibility options for devices running the Android operating system, with the emphasis on visually impaired users. We will analyze the features of the latest Android OS released, which is version 4.4, codename KitKat. It should be noted that there were no major updates to the accessibility options since Android 4.2.2 Jelly Bean.
\section{Speech Features}
Android also offers the option to interact with the device using speech and has some interesting accessibility features and compared to Windows Phone 8 offers a wider language support. Similarly to Windows Phone, an Android developer can take advantage of speech recognition and text-to-speech (TTS). Android comes with a number of built-in voice commands but unlike the Windows Phone, Android doesn’t allow developers to expose their own voice commands. The last important feature on Android is TalkBack.
At the time of writing, the speech recognition supports more than 40 languages including several accents of English, and even minor languages such as Czech. Other functions do not have such a wide support. 
\subsection{Speech Recognition}
Users can give input to an app or accomplish tasks with it using speech recognition. An example usage can be dictating content of an SMS. As mentioned before, this feature supports many languages but on the other hand required internet connection. There is not an option to use the recognition offline. We do not consider this a major drawback, as nowadays a mobile internet connection is more available than ever.
\subsection{Voice Action Commands}
In Android, Voice Action Commands are closely related to the Google Now feature. Google Now has a wide range of uses not specifically designed for visually impaired. It can, however, serve them well by allowing them to get information using voice.
In general, google now should provide the user with relevant information when they need it. Google describes it by the phrase “The right information at just the right time”. This includes telling the user the weather forecast, showing the best route to work, calling someone, creating a reminder and much more.

Note that for some commands, the system gives you a spoken answer. The current drawback of the system is that it only supports English, French, German, Spanish, and Italian. With other languages, user can only make a voice-induced Google search with no voice response.
Text to Speech (TTS)
TTS can be used to speak text to the user via the phone’s speaker or headset. The spoken text can be simple strings. The industry-standard Speech Synthesis Markup Language (SSML) is supported only in a limited scope. TTS is also used in TalkBack which is described in the next section.
\subsection{Other Speech Features}
TalkBack is an important functionality that strives for more accessible phone control for visually impaired. Basically, it is a touch-controlled screen reader. When enabled, user can drag finger across the screen selecting the components and getting their acoustic description. By double tapping anywhere in the screen, user can open/use the last selected item. TalkBack also supports gestures. This way, a user can get a complete description of the user interface. The blogpost of a blind accessibility engineer from Mozilla Foundation [3] claims that visually impaired users of this system still have to overcome some obstacles.

\section{Other Tools for Ease of Access}
Android too comes with more features for ease of access which can help lightly visually impaired users.
change font size in 
screen magnifier

\endinput

  %...

  %\startAppendices
  %  \chapter{Appendix}
Some introductory text\dots

\section{Section}
Another text\dots

\endinput
%%
%% End of file `app01.tex'.

  %\stopAppendices
\stopBodyMatter

\startBackMatter
  \PrintBibliography
  %\PrintIndex % define index entry in the text by: \index{word}
\stopBackMatter

\end{document}

\endinput
%%
%% End of file
