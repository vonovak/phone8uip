\chapter{iPhone Accessibility}
In this chapter, we will cover the accessibility of Apple’s iOS. Again we consider the latest iOS released, which is version 7.0.4. Overall, the accessibility features of iOS are very similar to those of Android and therefore I will describe the features more briefly.
\section{Speech Features}
As with the previous two platforms, iOS also offers users o interact with a device using speech. Apple was the first one to introduce the features for people with disabilities, such as VoiceOver. iOS supports speech recognition and text-to-speech in 15 major languages (the same number as Windows Phone 8).
iOs also comes with a number of built-in voice commands but doesn’t allow developers to expose their own voice commands.
\subsection{Speech Recognition - Dictation}
Users can give input to an app or accomplish tasks with it using speech recognition. An example usage can be dictating content of a text. As mentioned before, this feature supports 15 languages and requires an internet connection.
\subsection{Voice Control - Siri}
Siri in iOS can be thought of as an equivalent to Android’s Google Now. Siri can send emails, set reminders and more. If asked a question, it can read aloud the answer. 
\subsection{Text to Speech (TTS)}
TTS can be used to speak text to the user via the phone’s speaker or headset and this feature was added only recently, in iOS 7.0. The spoken text can be simple strings. The industry-standard Speech Synthesis Markup Language (SSML) is not mentioned in the API documentation.
\subsection{Other Speech Features}
It could be said that Google’s TalkBack is Apple’s VoiceOver. Both offer very similar functions and their key reason for existence is reading the content of the screen based on touch input and control of the device by gestures.
The mentioned blogpost of the blind accessibility engineer from Mozilla Foundation [3] favors VoiceOver over TalkBack.

\section{Other Tools for Ease of Access}
iOS too comes with more features for ease of access which can help lightly visually impaired users.
change font size 
Invert Colors
screen magnifier (Zoom)
Braille Displays for iOS
iPad, iPhone (3GS or later), and iPod touch (3rd generation or later) support more than 40 Bluetooth wireless braille displays right out of the box. Simply pair one and start using it to navigate your iOS device with VoiceOver — no additional software needed. In addition, iPad, iPhone, and iPod touch include braille tables for more than 25 languages.

\endinput